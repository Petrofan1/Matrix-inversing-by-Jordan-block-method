\documentclass[14pt,a4paper, openany]{article}
\usepackage[utf8]{inputenc}
\usepackage[english,russian]{babel}
\usepackage{amsthm,amsmath,amssymb,amsfonts,textcomp,latexsym,pb-diagram}
\usepackage{setspace}
\onehalfspacing
\usepackage{color}
\usepackage{listings}
\definecolor{codegreen}{rgb}{0,0.6,0}
\definecolor{codegray}{rgb}{0.5,0.5,0.5}
\definecolor{codepurple}{rgb}{0.58,0,0.82}
\definecolor{backcolour}{rgb}{0.95,0.95,0.92}
\usepackage{geometry} 
\geometry{tmargin=2cm,bmargin=2cm,lmargin=2cm,rmargin=1.5cm}
\renewcommand{\thesection}{\arabic{section}.}
\renewcommand{\thesubsection}{\arabic{section}.\arabic{subsection}}
\lstdefinestyle{mystyle}{
backgroundcolor=\color{backcolour},
commentstyle=\color{codegreen},
keywordstyle=\color{magenta},
numberstyle=\tiny\color{codegray},
stringstyle=\color{codepurple},
basicstyle=\ttfamily\footnotesize,
breakatwhitespace=false,
breaklines=true,
captionpos=b,
keepspaces=true,
numbers=left,
numbersep=5pt,
showspaces=false,
showstringspaces=false,
showtabs=false,
tabsize=2
}

\lstset{style=mystyle}
\begin{document}
\title{\textbf{Нахождение обратной матрицы методом Жордана с выбором главного элемента по строке. MPI алгоритм.}}
\author{Смирнов Георгий}
\date{310 группа}
\maketitle

\section{Введение}
\indent Дана матрица $A^{n \times n}$. Требуется найти обратную к ней матрицу $A^{-1}$, используя блочный метод Жордана с выбором главного элемента по строке с p процессами.
  Пусть $n = m \cdot l + s$. Тогда матрицу \textit{A} можно представить в виде:
$$
\left (
\begin{matrix} 
A_{11}^{m \times m} & A_{12}^{m \times m} & \ldots & A_{1l}^{m \times m} & A_{1,\ l+1}^{m \times s} \\
A_{21}^{m \times m} & A_{22}^{m \times m} & \ldots & A_{2l}^{m \times m} & A_{2,\ l+1}^{m \times s} \\
\vdots  & \vdots & \ddots & \vdots & \vdots \\
A_{l1}^{m \times m} & A_{l2}^{m \times m} & \ldots & A_{ll}^{m \times m} & A_{l, l+1}^{m \times s} \\
A_{l+1,\ 1}^{s \times m} & A_{l+1,\ 2}^{s \times m} & \ldots & A_{l+1,\ l}^{s \times m} & A_{l+1,\ l+1}^{s \times s}
\end{matrix}
\right )
$$ \\
Для нахождения обратной матрицы используем присоединённую матрицу $ B = E^{n \times n} $, разделённую на блоки аналогичным образом.
\section{Описание блочного метода Жордана с p процессами}
\subsection{Разделение на процессы.}
Теперь разделим строки матриц $A$ и $B$ между $p$ процессами. Процессу с номером $q\ (q\ \leq\ p)$ будут соответствовать строки с номерами
$q\ +\ p\cdot t\ (t \in \mathbb{N}\cup \{ 0\} ,\ q\ +\ p\cdot t \leq (l\ +\ 1))$.
\subsection{Шаги блочного метода Жордана с p процессами.}
\indent Все шаги метода Жордана будут описаны для произвольного $ k =\ 1 \ldots l $. Шаги для $ k = l+1 $ будут описаны отдельно. Начинаем с $ k = 1 $:
\begin{enumerate}
	\item Для начала разошлем всем процессам текущие $k$-е строки матриц $A$ и $B$ общим размером $2\cdot m\cdot n$ (здесь \textbf{первый} обмен данными). Затем распределим блоки $A_{kj}^{m \times m}$ ($ j = k \ldots l $) между процессами таким образом, как в начале распределили строки 
	(в силу того, что матрица квадратная и количество блоков в строке равно количеству блоков в столбце, нумерация столбцов согласно их процессам 
	соответствует аналогичной нумерации строк). После этого, в каждом процессе найдем обратные матрицы (обычным методом Жордана) для тех блоков, для которых это возможно. Среди них 
	в каждом процессе выберем блок с минимальной нормой\footnote{ В качестве нормы матрицы \textit{A} принимаем $ \|A^{m \times m} \| := \max\limits_{i = 1,\ldots,m} \sum\limits_{j = 1}^m |a_{ij}| $}. 
	Произведем обмен данными между процессами: например, с помощью MPI Allreduce. Если ни в одном из процессов не удалось найти обратимые блоки, метод не применим для заданного $ m $. Иначе, среди 
	уже найденных минимальных выберем в качестве главного элемента $A_{kk}^{m \times m}$ такой блок $A_{kj}^{m \times m}$, 
	что норма $ \|(A_{kj}^{m \times m})^{-1}\| $ будет минимальна для всех (здесь \textbf{второй} обмен данными). Далее, в каждом из процессов меняем $k$-й и $j$-й столбцы матрицы \textit{A} 
	(столбцы матрицы \textit{B} на данном этапе не меняем). 
	\item С помощью обычного метода Жордана в каждом процессе обращаем $ A_{kk}^{m \times m} $.
	\item 
	\textit{Этот шаг выполняется всеми процессами независимо.}
	Умножаем $k$-ю строку матрицы $A$ слева на $(A_{kk}^{m \times m})^{-1}$ (на первом шаге предварительно разослали всем процессам $k$-ю строку матрицы $A$):
	$$
	\begin{array}{l l}
		j =\ k+1 \ldots l:
		& \quad A_{kj}^{m \times m} \ \longrightarrow \ (A_{kk}^{m \times m})^{-1} \times A_{kj}^{m \times m} \\
		j = l+1:  &  \quad A_{k,\ l+1}^{m \times s} \ \longrightarrow \ (A_{kk}^{m \times m})^{-1} \times A_{k,\ l+1}^{m \times s}
	\end{array}
	$$
	Умножаем $k$-ю строку матрицы $B$ слева на $(A_{kk}^{m \times m})^{-1}$ (на первом шаге предварительно разослали всем процессам $k$-ю строку матрицы $B$):
	$$
	\begin{array}{l l}
		j =\ 1 \ldots l:
		& \quad B_{kj}^{m \times m} \ \longrightarrow \ (A_{kk}^{m \times m})^{-1} \times B_{kj}^{m \times m} \\
		j = l+1:  &  \quad B_{k,\ l+1}^{m \times s} \ \longrightarrow \ (A_{kk}^{m \times m})^{-1} \times B_{k,\ l+1}^{m \times s}
	\end{array}
	$$
	\item 
	\textit{Этот шаг выполняется всеми процессами независимо. Шаг описан для $q$-го процесса.}
	Для $i =\ q\ +\ p\cdot t\ (t \in \mathbb{N}\cup \{ 0\},\ q\ +\ p\cdot t\ <\ (l\ +\ 1),\ q\ +\ p\cdot t\ \neq k)$ в матрицах \textit{A} и \textit{B} из $i$-той строки 
	вычитаем $k$-ую, умноженную на $A _{ik}^{m \times m}$: \\
	$$
	\begin{array}{l l}
	 j =\ k+1 \ldots l:
	 & \quad A _{ij}^{m \times m} \longrightarrow (A _{ij}^{m \times m} - A _{ik}^{m \times m} \times  A _{kj}^{m \times m}) \\
	 j = l+1:  &  \quad A _{i,\ l+1}^{m \times s} \longrightarrow (A _{i,\ l+1}^{m \times s} - A _{ik}^{m \times m} \times  A _{k,\ l+1}^{m \times s})
	\end{array}
	$$
	$$
	\begin{array}{l l}
	j =\ 1 \ldots l:
	  	& \quad B _{ij}^{m \times m} \longrightarrow (B _{ij}^{m \times m} - A _{ik}^{m \times m} \times  B _{kj}^{m \times m}) \\
	j = l+1:  
		&  \quad B _{i,\ l+1}^{m \times s} \longrightarrow (B _{i,\ l+1}^{m \times s} - A _{ik}^{m \times m} \times  B _{k,\ l+1}^{m \times s})
	\end{array}
	$$
	Если $(l\ +\ 1)$-я строка попала в процесс с номером $q$:\\ 
	$$
	\begin{array}{l l l}
		i = l+1, &  j =\ k \ldots l:
	  & \quad A _{l+1,\ j}^{s \times m} \longrightarrow (A _{l+1,\ j}^{s \times m} - A _{l+1,\ k}^{s \times m} \times  A _{kj}^{m \times m}) \\
	 & j = l+1:  & \quad A _{l+1,\ l+1}^{s \times s} \longrightarrow (A _{l+1,\ l+1}^{s \times s} - A _{l+1,\ k}^{s \times m} \times  A _{k,\ l+1}^{m \times s})
	\end{array}
	$$
	$$
	\begin{array}{l l l}
	i = l+1, &  j =\ 1 \ldots l:
	  & \quad B _{l+1,\ j}^{s \times m} \longrightarrow (B _{l+1,\ j}^{s \times m} - A _{l+1,\ k}^{s \times m} \times  B _{kj}^{m \times m}) \\
	 & j = l+1:  & \quad B _{l+1,\ l+1}^{s \times s} \longrightarrow (B _{l+1,\ l+1}^{s \times s} - A _{l+1,\ k}^{s \times m} \times  B _{k,\ l+1}^{m \times s}) 
	\end{array}
	$$
	В локальной нумерации $i$ меняется немного иначе: $i\ =\ 1 \ldots r$, где $r$ колчество таких $t\in \mathbb{N}$, что 
	$q\ +\ p\cdot t\ \leq\ (l\ +\ 1)$. Учитываем, что если $k$-я строка принадлежит нашему процессу (т.е. существует такой 
	$t\in \mathbb{N}$, что $q\ +\ p\cdot t\ =\ k$), то соответствующий этой строке индекс $i$ в локальной нумерации 
	также нужно пропустить.

	\item Если $ k \neq l $ увеличиваем $k$ на 1 и переходим к 1-ому шагу. Иначе переходим на следующий шаг.\\\\\\
	\textit{\fbox{\textbf{Замечание.}} Дальнейшие действия описаны для $k$ = $l\ +\ 1$, т.е. для последнего шага алгоритма Жордана. Так же, как и раньше, в начале этого шага алгоритма Жордана рассылаем всем потокам $(l\ +\ 1)$-е строки матриц $A$ и $B$. 
	Далее, все процессы работают независимо друг от друга. Шаги описаны для процесса с номером $q$.}
	\item 
	С помощью обычного метода Жордана находим $ (A_{l+1,\ l+1}^{s \times s})^{-1} $
	Если блок необратим, то метод не применим для заданного $ m $.
	\item Умножаем $(l\ +\ 1)$-ю строку матрицы $B$ слева на $(A_{l+1,\ l+1}^{s \times s})^{-1}$:
	$$
	\begin{array}{l l}
		j = 1 \ldots l:	
			& \quad B_{l+1, j}^{s \times m} \ \longrightarrow \ (A_{l+1,\ l+1}^{s \times s})^{-1} \times B_{l+1,\ j}^{s \times m} \\
		j=l+1: 
			& \quad B_{l+1,\ l+1}^{s \times s} \ \longrightarrow \ (A_{l+1,\ l+1}^{s \times s})^{-1} \times B_{l+1,\ l+1}^{s \times s}
	\end{array}
	$$\\ \\ 
	\item Для $i =\ q\ +\ p\cdot t\ (t \in \mathbb{N}\cup \{ 0\},\ q\ +\ p\cdot t\ <\ (l\ +\ 1))$ в матрице \textit{B} из $i$-той строки вычитаем ($l+1$)-ую, умноженную на $A _{i,\ l+1}^{m \times s}$: \\
	$$
	\begin{array}{l l}
	j =\ 1 \ldots l:
	  & \quad B _{ij}^{m \times m} \longrightarrow (B _{ij}^{m \times m} - A _{i,\ l+1}^{m \times s} \times  B _{l+1,\ j}^{s \times m}) \\
	j = l+1:  &  \quad B _{i,\ l+1}^{m \times s} \longrightarrow (B _{i,\ l+1}^{m \times s} - A _{i,\ l+1}^{m \times s} \times  B _{l+1,\ l+1}^{s \times s}) \\
	\end{array}
	$$
	В локальной нумерации $i$ меняется немного иначе: $i\ =\ 1 \ldots r$, где $r$ колчество таких $t\in \mathbb{N}$, что 
	$q\ +\ p\cdot t\ <\ (l\ +\ 1)$.
	\item Выходим из алгоритма.
	\end{enumerate}

	\noindent\rule{\textwidth}{1pt}
	\indent В результате вышепредставленных шагов получим следующую матрицу:\\
	$$
	\left (
	\begin{matrix} 
	E_{11}^{m \times m} & 0 & \ldots & 0 & 0 \\
	0 & E_{22}^{m \times m} & \ldots & 0 & 0 \\
	\vdots  & \vdots & \ddots & \vdots & \vdots \\
	0 & 0 & \ldots & E_{ll}^{m \times m} & 0 \\
	0 & 0 & \ldots & 0 & E_{l+1,\ l+1}^{s \times s}
	\end{matrix}
	\right .
	\left |
	\begin{matrix} 
	\tilde B_{11}^{m \times m} & \tilde B_{12}^{m \times m} & \ldots & \tilde B_{1l}^{m \times m} & \tilde B_{1,\ l+1}^{m \times s} \\
	\tilde B_{21}^{m \times m} & \tilde B_{22}^{m \times m} & \ldots & \tilde B_{2l}^{m \times m} & \tilde B_{2,\ l+1}^{m \times s} \\
	\vdots  & \vdots & \ddots & \vdots & \vdots \\
	\tilde B_{l1}^{m \times m} & \tilde B_{l2}^{m \times m} & \ldots & \tilde B_{ll}^{m \times m} & \tilde B_{l,\ l+1}^{m \times s} \\
	\tilde B_{l+1,\ 1}^{s \times m} & \tilde B_{l+1,\ 2}^{s \times m} & \ldots & \tilde B_{l+1,\ l}^{s \times m} & \tilde B_{l+1,\ l+1}^{s \times s}
	\end{matrix}
	\right )
	$$\\
	\indent Если бы мы искали матрицу, обратную к матрице $A$, методом Жордана с выбором главного элемента по столбцу, переставляя только строки матрицы $A$, в правой части мы получили бы обратную матрицу. Однако в методе Жордана с выбором главного элемента по строке в матрице $A$ мы переставляем не строки, а столбцы. Пусть $U_1, U_2,\ldots, U_l$ - элементарные матрицы, соответствующие перестановкам столбцов. Тогда перестановки столбцов меняют матрицу следующим образом:\\ \\
	$A \longrightarrow AU_1\ldots U_l$\\ \\
	Так как $AU_1\ldots U_l = E$, то
	$(AU_1\ldots U_l)^{-1} = U_l^{-1}\ldots U_1^{-1}A^{-1} = E$.
	Откуда обратная матрица равна $A^{-1} = U_1\ldots U_l$. 
	Следовательно, чтобы получить ответ, необходимо поменять у матрицы $\tilde B$ строки так, как у матрицы $A$ менялись столбцы: \\ \\
	$A^{-1} = U_1\ldots U_l\tilde B$.
	\subsection{Обмен данными.}
	Все обмены между данными были отмечены в течение описания алгоритма. Всего их $2\cdot(l\ +\ 1)$ штук. Объем каждого обмена $2\cdot n\cdot m$.	\subsection{Оценка числа операций для одного процесса.}
	\begin{enumerate}
		\item Для нахождения обратной $m\times m$ матрицы обычным методом Жордана требуется порядка $ 3m^3\ -\ \frac{m^2}{2}\ -\frac{m}{2}$ операций.
		\item Количество операций для умножения двух матриц $m\times m$:  $m\cdot m\cdot(m\ +\ m\ -\ 1) = 2m^3\ -\ m^2$.
		\item Количество операций для умножения двух матриц $m\times s$ и $s\times m$:  $m\cdot m\cdot(s+s-1) = 2m^2s - m^2$.
		\item Количество операций для умножения двух матриц $m\times m$ и $m\times s$: $m\cdot s\cdot(m\ +\ m\ -\ 1) = 2m^2s\ -\ ms$.
		\item Количество операций при реализации умножения блоков в пунктах 3 и 7: $(\sum\limits_{i=1}^{l}(l\ +\ l\ -\ i)\cdot (2m^3\ -\ m^2))\ +\ (l + l)\cdot (2m^2s\ -\ ms)\ +\ l\cdot (2s^2m\ - ms)\ +\ 2s^3\ -\ s^2\ =\ (\sum\limits_{i=1}^{l}(2l\ -\ i)\cdot (2m^3\ -\ m^2))\ +\ 2l\cdot (2m^2s\ -\ ms)\ +\ l\cdot (2s^2m\ - ms)\ +\ 2s^3\ -\ s^2\ $
		\item Не ограничивая общности, считаем, что $p\ |\ l$, так как иначе в качестве $l$ всегда можем взять такое $\tilde l\ \in \mathbb{N},\ \tilde l\ >\ l$, что $p\ |\ \tilde l$.
		Тогда, каждому процессу принадлежит $i\ =\ \frac{l}{p}$ строк.
		\item Количество операций для всех обращений в одном процессе порядка: $(3m^3\ -\ \frac{m^2}{2}\ -\frac{m}{2})\cdot \frac {(l^2\ +\ lp)}{2p}$
		\item Количество операций при реализации пунктов 4 и 8 в каждом из процессов: $(\sum\limits_{i=1}^{l}(l - 1)\cdot ((2l\ -\ p\cdot [\frac{i}{p}])\cdot 2m^3\ +\ 4lm^2)\ +\ 2(2l\ -\ p\cdot [\frac{i}{p}])\cdot lm^2)$
		\item Суммируем результаты, получаем: $\frac{m^3}{p}\cdot (3l^3\ +\ 0.5(l^2\ +\ 3l))\ +\ \frac{m^2}{p}\cdot (8sl^3\ -\ 3sl^2 +\ 6ls)\ +\ \frac{m}{p}\cdot (8ls^2\ -\ 0.5l^2\ -\ 0.25l\ -\ 3ls)$ 
		\item В итоге, для $\ p\ =\ 1$: $m^3\cdot (3l^3\ +\ 0.5(l^2\ +\ 3l))\ +\ m^2\cdot (8sl^3\ -\ 3sl^2 +\ 6ls)\ +\ m \cdot (8ls^2\ -\ 0.5l^2\ -\ 0.25l\ -\ 3ls)$  
	\end{enumerate}
\end{document}

